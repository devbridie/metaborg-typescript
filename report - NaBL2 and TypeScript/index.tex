%% For double-blind review submission
\documentclass[sigplan,10pt]{acmart}\settopmatter{printfolios=true}
%% For single-blind review submission
%\documentclass[sigplan,10pt,review]{acmart}\settopmatter{printfolios=true}
%% For final camera-ready submission
%\documentclass[sigplan,10pt]{acmart}\settopmatter{}

%% Note: Authors migrating a paper from traditional SIGPLAN
%% proceedings format to PACMPL format should change 'sigplan' to
%% 'acmsmall'.

\usepackage[T1]{fontenc}
\usepackage{lmodern}
%% Some recommended packages.
\usepackage{booktabs}   %% For formal tables:
                        %% http://ctan.org/pkg/booktabs
\usepackage{subcaption} %% For complex figures with subfigures/subcaptions
                        %% http://ctan.org/pkg/subcaption
\usepackage{listings}
\usepackage{float}
\usepackage{dblfloatfix}
\usepackage{placeins}

\lstset{
frame = single,
basicstyle=\small
}

%% Copyright information
%% Supplied to authors (based on authors' rights management selection;
%% see authors.acm.org) by publisher for camera-ready submission
\setcopyright{none}             %% For review submission
%\setcopyright{acmcopyright}
%\setcopyright{acmlicensed}
%\setcopyright{rightsretained}
%\copyrightyear{2017}           %% If different from \acmYear


%% Bibliography style
\bibliographystyle{ACM-Reference-Format}
%% Citation style
%% Note: author/year citations are required for papers published as an
%% issue of PACMPL.
%\citestyle{acmauthoryear}  %% For author/year citations
%\citestyle{acmnumeric}     %% For numeric citations
%\setcitestyle{nosort}      %% With 'acmnumeric', to disable automatic
                            %% sorting of references within a single citation;
                            %% e.g., \cite{Smith99,Carpenter05,Baker12}
                            %% rendered as [14,5,2] rather than [2,5,14].
%\setcitesyle{nocompress}   %% With 'acmnumeric', to disable automatic
                            %% compression of sequential references within a
                            %% single citation;
                            %% e.g., \cite{Baker12,Baker14,Baker16}
                            %% rendered as [2,3,4] rather than [2-4].



\begin{document}

%% Title information
\title[Metaborg Typescript]{Implementing Typescript in Metaborg Spoofax}         %% [Short Title] is optional;
                                        %% when present, will be used in
                                        %% header instead of Full Title.
%% Author information
%% Contents and number of authors suppressed with 'anonymous'.
%% Each author should be introduced by \author, followed by
%% \authornote (optional), \orcid (optional), \affiliation, and
%% \email.
%% An author may have multiple affiliations and/or emails; repeat the
%% appropriate command.
%% Many elements are not rendered, but should be provided for metadata
%% extraction tools.

%% Author with single affiliation.
\author{Tim van der Lippe}

%% Author with two affiliations and emails.
\author{Thomas Smith}

%% Abstract
%% Note: \begin{abstract}...\end{abstract} environment must come
%% before \maketitle command
\begin{abstract}
TypeScript is a structurally typed superset of JavaScript that aims to ease the introduction of static types
to large JavaScript code bases. In this paper, we investigate how the static semantics of TypeScript can be
modeled using NaBL2. We find that while NaBL2 misses some features needed to support structural types, 
these features can be implemented in terms of the existing abstractions that NaBL2 is built on.
We show that Scope Graphs can be used to model structural types and propose a minimal set of extensions
to NaBL2 that enable language designers to define custom subtype relations. We also show that these extensions
are enough to implement most major features of the TypeScript language.
\end{abstract}

%% Keywords
%% comma separated list
\keywords{Metaborg, Spoofax, Typescript, NaBL2}  %% \keywords is optional


%% \maketitle
%% Note: \maketitle command must come after title commands, author
%% commands, abstract environment, Computing Classification System
%% environment and commands, and keywords command.
\maketitle


\section{Introduction}

TypeScript\footnote{https://github.com/Microsoft/TypeScript/blob/master/doc/spec.md} is a statically typed programming language that is a superset JavaScript (ECMAScript 2015\footnote{http://www.ecma-international.org/ecma-262/6.0/}).
The aim of TypeScript is to provide large JavaScript projects with a way to integrate a static 
type system into their code in a non-intrusive way, 
while keeping the dynamic semantics equivalent to JavaScipt.
To achieve this, TypeScript compiles to human-readable JavaScript with as little transformations as possible.
Any value that has type annotations is checked by the type checker.
In addition, some type inference is done to provide static checks even for existing JavaScript without annotations.
Only when no type information can be inferred will the compiler fall back to default JavaScipt semantics.
The system of structural types enables TypeScript to be non-intrusive when used with existing JavaScript projects.
It allows programmers to gain some of benefits of static analysis while still 
being able to use existing JavaScript APIs, libraries and coding conventions.

In this report we explore how structural type systems could be implemented in terms of Scope 
Graphs with NaBL2\citep{Antwerpen:2016:CLS:2847538.2847543}.
We find that while NaBL2 already works very well for nominally typed languages with classes and inheritance,
structural types are not yet fully supported. 

Our main contribution is the proposal of a minimal set of extensions to NaBL2 that 
give language designers the possibility to define their own subtype relations. 
We show that the fundamental model of Scope Graphs is viable even for structural types, 
and that making constraint generation and resolution interleaved
allows language designers to model more complex and diverse type systems with NaBL2.

An introduction to structural type systems is given in Section \ref{sec:structural-typing}.
Then, the most significant features of the TypeScript language are discussed in Section \ref{sec:structural-typing-typescript}.
After the introduction of some of TypeScripts features, we move on to discuss, in Section \ref{sec:syntax}, how SDF3 could be used to 
easilly mirror the ECMAScript 2015 specification with the additional syntactical constructs of TypeScript.
Finally, Section \ref{sec:type-checking} discusses how structural typing can be supported in NaBL2 
as well as possible implementations of some TypeScript features in NaBL2.

\section{Structural Typing}
\label{sec:structural-typing}

This section gives a high level overview of how structural type systems work.
Structural subtyping stands opposed to nominal subtyping. 
A majority of the most used programming languages employ a nominal type system.
In such languages, subtype relationships are explicitly declared.
The programmer has to relate two types to each other by their name,
after which the compiler will check whether the structures refered to by these names are compatible.
This process has to be done only once for each declared subtype relation.
As soon as the compiler has established that a type A is a subtype of B,
the next time that a check for this relationship has to be performed, 
a simple table lookup for the two names is sufficient.
\bigskip
With structural subtyping on the other hand, the relationships between types are implicit.
The programmer does not have to specify any relationship by hand,
every time the compiler has to check compatibility between two types, a structural comparison of the types is done.
One major consequence of this approach is that names for types become purely cosmetic.
The programmer can still assign names to types, and refer to those types by their names,
but checking whether one type is a subtype of another is completely dependent on their structure.
Generally this means that a type A is a subtype of B when the structure of A has at least as much information as the strucrure of B.
This means that a value of type A can be assigned to B by simply removing all information present in A that is not required by B.
For most languages that implement structural subtyping, the information that types carry consists of names paired with another type.
For example, in TypeScript we may write:


\begin{lstlisting}
type A = { first: string, second: string }
\end{lstlisting}

This defines the type A to be an object that contains two named fields of type 'string'.
Values of type A can be passed to anywhere a subset of \texttt{\{ first: string, second: string \}} is required.
So all the following statements are valid because the types of $x$, $y$ and $z$ are structurally subsets of $A$:

\begin{lstlisting}
const a: A =  { first: 'x', second: 'y' }

const x: { first: string } = a
const y: { second: string } = a
const z: {} = a
\end{lstlisting}

The nature of such type systems makes implementing a type checker slightly more difficult than with a nominal type system.
To demonstrate this, we consider recursive types:
\begin{lstlisting}
type Nil = {}
type Cons = { head: number, tail: NumList }
type NumList = Cons | Nil
\end{lstlisting}

The \texttt{|} operator describes that $NumList$ can either be $Cons$ nil $Nil$.
Note that $Cons$ and $NumList$ are mutually recursive.
To check whether we can assign a $NumList$ value to another value, 
we need to check if their unfoldings are equal.
Since unfolding a recursive type yields an infinite tree, 
the type checker has to make sure that this process terminates regardless of the possibly infinite type.
This approach to type checking is called \textit{equi-recursive} because a recursive type should
be a subtype of its one step unfolding\citep{pierce2002types}.

\section{Structural Typing in Typescript}
\label{sec:structural-typing-typescript}
TypeScript was built with structural typing as its backbone, 
nearly any feature in the language can be described in terms of structual types.
The reason for this being that the goal of the language is to provide a type system
on top of the dynamically typed JavaScript language while not introducing any 
abstractions that have a conceptually high distance to the semantics of JavaScript.
The result of this effort means that compiled TypeScript code should be suitable 
for human consumption. Structural typing provides an excellent bridge from the 
dynamically typed JavaScript world to the statically typed world of TypeScript.
In this section we will introduce the most important features of TypeScript and
how they relate to structural types.
\bigskip
Structural types manifest themselves in TypeScript as \textit{Record Types}.
In the examples above, we have already encountered some record types, a set of name-type pairs.
Built directly on top of record types are some of the most important aspects of the language:
\begin{itemize}
\item classes
\item interfaces
\item union types
\item intersection types
\end{itemize}

Other notable features include: first class and higher order functions, parametric and ad-hoc polymorphism 
(generics) and several types for special values like \texttt{null} and \texttt{undefined}. It is 
worth noting that the subtype relation within TypeScript does not form a proper latice due to 
the \texttt{any} type. Namely, a value of type \texttt{any} is a subtype of every other type, 
while every type is simultaniously a subtype of any.
\bigskip
\subsection{Classes and Interfaces}
While syntax for defining classes in TypeScript is similar to Java, the structural typing makes classes semantically different form nominally typed languages.
The following snippet defines a simple class in TypeScript:

\begin{lstlisting}
class NonEmptyList<A> {
  head: A;
  tail: List<A>;
  constructor(head: A) {
    this.head = head;
    this.tail = new List();
  }
}
\end{lstlisting}

In any nominally typed language, it would not be possible to assign a \texttt{NonEmptyList} to any variable that is not also a \texttt{NonEmptyList}.
In TypeScript this restriction does not apply. 
Any instance of a class can be assigned a variable with any type as long as the two are structurally in a subtype relationship.
This means that the following snippet is perfectly legal:

\begin{lstlisting}
class SignletonList<A> {
  head: A;
}

const s: SingletonList<number> = 
  new NonEmptyList(42);
\end{lstlisting}

Since names are irrelevant to the type checker, the following is also valid because the type of 
\texttt{s} is structurally equivalent to \texttt{NonEmptyList<number>}.
\bigskip
\begin{lstlisting}
const s: { 
  head: number, tail: List<number> 
} = new NonEmptyList(42);
\end{lstlisting}

This situation remains unchanged when we add 'methods' to the \texttt{NonEmptyList} class:

\begin{lstlisting}
class NonEmptyList<A> {
  head: A;
  tail: List<A>;
  ...
  size(): number {
    return 1 + tail.size();
  }
}

const s: { 
  head: number, tail: List<number>,
  size: () => number  
} = new NonEmptyList(42);
\end{lstlisting}

From the above snippet we can see that classes are simply sugar for named object types.
The fields are directly translated to object members, and class methods are translated
to object members that have a function type corresponding to the method signature.
A consequence of this is that class methods are first class values.
Class constructors are translated to a special type of function outside the class scope that has the exact name of the class:

\begin{lstlisting}
class NonEmptyList<A> {
  ...
  constructor(head: A) {
    ...
  }
}
const a: typeof NonEmptyList = NonEmptyList
const xs: NonEmptyList<number> = new a(42)
\end{lstlisting}

From this it appears that classes live in both the value namespace as well as the type namespace,
because an instance of \texttt{NonEmptyList} ($xs$) can be created from a runtime value ($a$) on the last line.
In reality, $a$ represents the constructor for \texttt{NonEmptyList} rather than the class itself.
We can therefore conclude that class constructors are first class values just like class methods.
The type of a constructor must be different from the type of a function because the former can only be called with the \texttt{new} keyword.
The \texttt{typeof} keyword in the definition of $a$ is used in this case to signal that $a$ is a constructor.
Since constructors do not live inside the class scope, they have no impact on type checking when handling instances of a class.
\bigskip
Inheritance is also handled in a straightforward manner.
When checking whether an instance of a class can be assigned to an object type, the compiler
simply takes all class members and all members of any parent class into account and checks that those are a subset of the assinged value.
The rules that apply to classes can similarly be applied to interfaces, with the exception that interfaces do not have a constructor.

\subsection{Union and Intersection types}
Union and intersection types provide for a way to compose types, similar to sum and product types in many functional languages.
\begin{lstlisting}
interface X { x: number }
interface Y { y: string }
type U = X | Y
\end{lstlisting}

The last line in the code above defines $U$ to be the union of $X$ and $Y$. 
This means that $U$ can be either $X$ or $Y$ but not both, similar to discriminated unions in other languages.
Union types have the following subtype relationships:
\begin{itemize}
\item A union type $U$ is a subtype of a type $T$ if each type in $U$ is a subtype of $T$.
\item A type $T$ is a subtype of a union type $U$ if $T$ is a subtype of any type in $U$.
\end{itemize}
Similarly, union types have the following assignability relationships:
\begin{itemize}
\item A union type $U$ is assignable to a type $T$ if each type in $U$ is assignable to $T$.
\item A type $T$ is assignable to a union type $U$ if $T$ is assignable to any type in $U$.
\end{itemize}

When the types that make up the union have overlapping members, those members themselves become unions:
\begin{lstlisting}
interface X { q: number }
interface Y { q: string }
type U = X | Y
// U is equal to { q: number | string }
\end{lstlisting}
\bigskip
Intersection types describe values with multiple simultanious types. 
\begin{lstlisting}
interface X { x: number }
interface Y { y: string }
type I = X & Y
// I is equal to { x: number, y: string }
\end{lstlisting}

Just as with unions, when the types that make up an intersection have overlapping members, those members themselves become intersections:
\begin{lstlisting}
interface X { q: number }
interface Y { q: string }
type I = X & Y
// I is equal to { q: number & string }
\end{lstlisting}

Intersection types have the following subtype relationships:
\begin{itemize}
\item An intersection type $I$ is a subtype of a type $T$ if any type in $I$ is a subtype of $T$.
\item A type $T$ is a subtype of an intersection type $I$ if $T$ is a subtype of each type in $I$.
\end{itemize}
Similarly, intersection types have the following assignability relationships:
\begin{itemize}
\item An intersection type $I$ is assignable to a type $T$ if any type in $I$ is assignable to $T$.
\item A type $T$ is assignable to an intersection type $I$ if $T$ is assignable to each type in $I$.
\end{itemize}
\bigskip
Since functions are values in TypeScript, we can even define unions and intersections over functions.
The following example demonstrates the intersting case of an intersection over functions:
\begin{lstlisting}
type F1 = (a: string, b: string) => void;  
type F2 = (a: number, b: number) => void;

var f: F1 & F2 = (a, b) => { };  
f("hello", "world");
f(1, 2);
f(1, "test");
\end{lstlisting}

The last call to $f$ will be rejected by the compiler.
The type signature of $f$ makes it possible for the type checker to infer
that this function should only be called with two strings or two numbers.
If the type signature were removed, this program would be well-typed.
\bigskip
In the next sections we will discuss how these features would ideally be implemented in Spoofax.


\section{Syntax definition}
\label{sec:syntax}
The syntax definition of Typescript is defined in the TypeScript Language Specification.
Even though this specification is not up-to-date for TypeScript 2, the syntax has not significantly changed between TypeScript 1 and 2.
The specification builds upon the ECMAScript 2015 Language Specification.
While it is defined as a superset with additions to the syntax definitions, several productions in the TypeScript grammar modify or replace the productions with the same name in the ECMAScript definition.

The translation of the syntax definition into the SDF3\citep{Vollebregt:2012:DST:2427048.2427056} productions starts with the ECMAScript productions and is then updated with the TypeScript productions.
All SDF3 production symbol names are taken from the language specification(s), while the constructor labels are either a copy of the symbol name or a self-invented name.
An example of a translation of \textit{IfStatement} is shown in Figure \ref{fig:if-statement}.
For multi-line productions, whitespace and indentation is inserted based on personal preferences of the authors.
The SDF3 templates adhere to the indentation to achieve formatted code completion in the editor, which the language specification does not.
The ECMAScript language specification also includes parameters, denoted by a suffix in subscript, which are disregarded in the SDF3 production notation.

\begin{figure*}
  \begin{lstlisting}[caption=Definition of \textit{IfStatement} in the ECMAScript language specification,mathescape]
IfStatement$\textsubscript{[Yield, Return]}$ : See 13.6
  if ( Expression$\textsubscript{[In, ?Yield]}$ ) Statement$\textsubscript{[?Yield, ?Return]}$ else Statement$\textsubscript{[?Yield, ?Return]}$
  if ( Expression$\textsubscript{[In, ?Yield]}$ ) Statement$\textsubscript{[?Yield, ?Return]}$
  \end{lstlisting}
  \begin{lstlisting}[caption=Definition of \textit{IfStatement} in SDF3 production notation]
IfStatement.IfElse = <
  if (<PrimaryExpression>) <Statement>
  else <Statement>
>
IfStatement.If = <if (<PrimaryExpression>) <Statement>>
  \end{lstlisting}
  \caption{The translation of \textit{IfStatement} from the ECMAScript language specification to SDF3 production notation.}
  \label{fig:if-statement}
\end{figure*}

Some productions in the ECMAScript only directly inject other productions.
For these productions, the SDF3 notation does not define a constructor label, unless the label is required in the type-checking phase to prevent ambiguation on the constructor labels in the constraint generation phase.
An example of a production with only directly injected other productions is shown in Figure \ref{fig:direct-production-spec} and \ref{fig:direct-production-sdf}.
In the example, the extra constructor label \textit{ModuleStatement} is defined, to prevent disambiguation in programs and modules.

\begin{figure}[H]
\begin{lstlisting}
ModuleItem : See 15.2
  ImportDeclaration
  ExportDeclaration
  StatementListItem
\end{lstlisting}
\caption{Definition of \textit{ModuleItem} in the ECMAScript language specification}
\label{fig:direct-production-spec}
\end{figure}

\begin{figure}[H]
\begin{lstlisting}
ModuleItem = ImportDeclaration
ModuleItem = ExportDeclaration
ModuleItem.ModuleStatement = 
  StatementListItem
\end{lstlisting}
\caption{Definition of \textit{ModuleItem} in SDF3}
\label{fig:direct-production-sdf}
\end{figure}


% \begin{figure*}
%   \begin{minipage}[t]{.8\columnwidth}
%     \begin{lstlisting}[caption=Definition of \textit{ModuleItem} in the ECMAScript language specification]
% ModuleItem : See 15.2
%   ImportDeclaration
%   ExportDeclaration
%   StatementListItem
%     \end{lstlisting}
%   \end{minipage}
%   \hfill
%   \begin{minipage}[t]{1.1\columnwidth}
%     \begin{lstlisting}[caption=Definition of \textit{ModuleItem} in SDF3 production notation]
% ModuleItem = ImportDeclaration
% ModuleItem = ExportDeclaration
% ModuleItem.ModuleStatement = StatementListItem
%     \end{lstlisting}
%   \end{minipage}
%   \caption{The translation of \textit{ModuleItem} from the ECMAScript language specification to SDF3 production notation.}
%   \label{fig:direct-production}
% \end{figure*}

Lastly, Figure \ref{fig:override-syntax} shows a production that is modified by TypeScript and thus overrides the original definition from the ECMAScript language specification.
Note that the second production as defined in the TypeScript language definition has not been implemented in SDF3 production notation at the time of writing.

\begin{figure*}
  \begin{lstlisting}[caption=Definition of \textit{FunctionDeclaration} in the ECMAScript language specification,mathescape]
FunctionDeclaration$\textsubscript{[Yield, Default]}$ : See 14.1
  function BindingIdentifier$\textsubscript{[?Yield]}$ ( FormalParameters ) { FunctionBody }
  [+Default] function ( FormalParameters ) { FunctionBody }
  \end{lstlisting}
  \begin{lstlisting}[caption=Definition of \textit{FunctionDeclaration} in the TypeScript language specification,mathescape]
FunctionDeclaration: ( Modified )
  function BindingIdentifier$\textsubscript{opt}$ CallSignature { FunctionBody }
  function BindingIdentifier$\textsubscript{opt}$ CallSignature ;
  \end{lstlisting}
  \begin{lstlisting}[caption=Definition of \textit{FunctionDeclaration} in SDF3 production notation]
FunctionDeclaration.Function = <
function <BindingIdentifier?><CallSignature> {
  <FunctionBody>
}
>
  \end{lstlisting}
  \caption{The translation of \textit{FunctionDeclaration} from both the ECMAScript and TypeScript language specifications to SDF3 production notation.}
  \label{fig:override-syntax}
\end{figure*}

Disambiguation based on context-free priorities are encoded in the production hierarchy of the ECMAScript language specification.
The primary target for disambiguation is \textit{PrimaryExpression}.
The production rule in SDF3 includes all \textit{PrimaryExpression}, while the ECMAScript defines a strict ordering of the production rules in the production hierarchy.
At the time of writing, no explicit \textit{context-free-priorities} have been defined in SDF3.
In the examples implemented in the language project, no ambiguation has been detected thus far.

\newpage
\section{Type Checking}
\label{sec:type-checking}

In this section we will dicuss our attempt at using NaBL2 \citep{Antwerpen:2016:CLS:2847538.2847543} to implement the type system as explained in section \ref{sec:structural-typing-typescript}.
NaBL2 is a constraint generation language specifically tailored towards the problem of name binding.
The language allows language designers to specify constraints over AST constructs, which are then solved in one pass by the runtime.
A general solution to the name binding problem is provided by a framework of scope graphs.
A scope graph consists of scopes, declarations and references to names in a program.
Constraints can be generated to construct such graphs, as well as to resolve paths from declarations to references in the graph.
Since the problem of type checking is very similar to that of name binding, NaBL2 can also be used to implement a type checker.

\subsection{Structural type checking}

In general, NaBL2 is well-suited for statements and expressions that are common in programming languages, 
such as the shown \textit{IfStatement}.
During this project, our research focussed on exploring to what extent NaBL2 supports structural type checking and, 
in the case that it does not, what would be required to support it.
Our syntax definition for TypeScript includes several language constructs that must participate in type checking.
These constructs included: object declarations, variable assignments, interfaces, functions, function calls and primitives.
Several test cases and examples were developed that tested various aspects of structural type checking with these constructs.

To demonstrate our attempt at implementing structural types, we will use object assignment as example.
The code below shows a TypeScript program that should be rejected by the type checker 
because \texttt{Person} is a structural subtype of \texttt{Developer}. 
The assignment of \texttt{john} to a \texttt{Developer} on the last line is therefore illegal.
\begin{lstlisting}
type Person = {
  name: string
}
type Developer = {
  name: string
  language: string
}
var john: Person = { name: 'John' }
var devJohn: Developer = john
\end{lstlisting}

\subsection{Naive solution}
Object types can be modeled cleanly using scope graphs. 
The general approach is to define a type, \texttt{RECORD}, that wraps a \texttt{scope}.
The wrapped scope should contain the declarations of each field in the object type.
Checking whether two types are in a subtype relation then becomes the problem of checking whether
the the scope of the left-hand side \texttt{RECORD} type is a subset of the one on the right-hand side.
NaBL2 already supports scope subset checking.
This solution is shown in Figure \ref{fig:wrong-solution}.
The top constraint generation rule traverses over the \texttt{ObjectType} AST node.
This node corresponds to a type annotation on any variable or function argument in TypeScript.
The second generation rule traverses over variable declarations.
The second component of a \texttt{VariableDeclaration} is the declared type, 
while the third component denotes the value assigned to the variable.
Notice that \texttt{subseteq} is used to check that the left-hand side is a subset of the right-hand side.
Running this code on the example above produces the desired result.
NaBL2 will reject the last line based on the subset check.
\begin{figure*}[h]
\begin{lstlisting}
[[ ObjectType(fields) ^ (s): ty ]] :=
    new record_scope,
    ty == RECORD(record_scope),
    record_scope -P-> s,
    distinct/name D(record_scope)/Field,
    Map1 [[ fields ^ (record_scope) ]].

[[ VariableDeclaration(name, type@Id(_), value) ^ (s) ]] :=
    [[ type ^ (s) : ty ]],
    ty == RECORD(type_scope),
    [[ value ^ (s) : RECORD(value_scope) ]],
    D(type_scope)/Field subseteq/name D(value_scope)/Field,
    Value{name} <- s,
    Value{name} : ty !.
\end{lstlisting}
\caption{Naive solution to object type checking.}
\label{fig:wrong-solution}
\end{figure*}

While this solution is straightforward, there are two problems that prevent it from being useful.
First, a pattern match on \texttt{ type@Id(\_) } is needed to dertermine that the left-hand side is indeed an object type. 
This means that separate constraint generation rules are needed for checking non-object types and aliases to non-object types, leading to code duplication.
Second, the \texttt{subseteq} operator only checks whether names of fields in both scopes are in a subset relation.
Types that are assigned to these fields are completely ignored.
This means that we could assign any object to a \texttt{Person} that has a \texttt{name} field, 
regardless of the type of that field.
The \texttt{subseteq} operator is insufficient to solve structural type checking.

\subsection{Extensible subtype relations}
The two problems illustrated above can be solved by making the NaBL2 subtype check user extensible.
Normally, checking compatibility of two types in NaBL2 is done by generating constraints for equality or subtype checks using \texttt{==} or \texttt{<?}.
In an ideal scenario, we would be able to use the \texttt{<?} operator to generate constraints for structural subtyping.
However, this would require the operator to perform different checks based on the types of its arguments.
For non-object types like strings and numbers, a simple non-recursive check is sufficient.
Structural types require a recursive check. 
As explained in section \ref{sec:structural-typing}, checking whether two object types in an equi-recursive system 
(like with TypeScript) are subtypes requires recursively unfolding both types and checking their equality in the fixed point.
Allowing users of NaBL2 to overload the \texttt{<?} operator requires constraint generation and solving to be interleaved.
At solve-time, when a \texttt{<?} check is encountered, the solver must pause and evaluate the user-defined subtype relation, 
which in turn generates more constraints to be solved.
These constraints may include more, recursive, subtype checks in the case of recursive structural types like lists, trees, etc.
When custom subtype relations are allowed, 
the incorrect solution in Figure \ref{fig:wrong-solution} could be fixed by modifying it as shown in Figure \ref{fig:cool-solution}.
Notice that the pattern match is gone and the rule is completely polymorphic in the types of the left- and right-hand side.
\begin{figure*}
\begin{lstlisting}
[[ ObjectType(fields) ^ (s): ty ]] :=
    new record_scope,
    ty == RECORD(record_scope),
    record_scope -P-> s,
    distinct/name D(record_scope)/Field,
    Map1 [[ fields ^ (record_scope) ]].

[[ VariableDeclaration(name, type, value) ^ (s) ]] :=
    [[ type ^ (s) : ty ]],
    [[ value ^ (s) : valueTy ]],
    ty <? valueTy
    Value{name} <- s,
    Value{name} : ty!.
\end{lstlisting}
\caption{General solution when given the possibility of user-defined subtype checks.}
\label{fig:cool-solution}
\end{figure*}
In the next section, the syntactical details of defining custom subtype relations are discussed.

\subsection{NaBL2 syntax proposal}
To syntactically support the definition of custom subtype relations, 
we propose an extension to the syntax definition of NaBL2 as shown in Figure \ref{fig:nabl-syntax-proposal-relation}.
Each NaBL2 module can contain a \texttt{relations} section.
We propose adding a new relation pattern with two elements: 
\begin{itemize}
\item the left- and right-hand side of the subtype relation, denoted with common tuple-syntax
\item the constraint generated for these two patterns.
\end{itemize}
Note that this relation pattern supports one constraint, 
since multiple constraint can easilly be composed into one.
We also propose two new constraint constructs: \texttt{forall} and \texttt{exists} as shown in Figure \ref{fig:nabl-syntax-proposal-constraint}.
These constructs can be used to check that a constraint holds over all declarations in a scope, 
or that there exists a declaration in a scope that satisfies a constraint.
These three new syntactical constructs can be used to define a subtype relation for the \texttt{RECORD} type,
illustrated in Figure \ref{fig:nabl-syntax-proposal-usage}.
The \texttt{forall} will generate the constraint after the \texttt{=>} symbol for each declaration in the \texttt{one} scope.
The \texttt{exists} construct ensures that \texttt{ty1 <? ty2} holds for at least one declaration in the \texttt{two} scope.
The \texttt{sub : Type * Type} annotation on the second line determines that the name of the relation is \texttt{sub},
which can be referenced in constraint generation rules as either \texttt{<sub?} or \texttt{<?}.
Different relations could be made for different purposes, and each relation can be overloaded for multiple types of arguments.
For example, as shown in the last lines of \ref{fig:nabl-syntax-proposal-usage}, one could define another subtype relation over \texttt{List} types.
Therefore, this syntax is not restricted to structural types and can also be useful when designing other kinds of type systems.
The additional \texttt{forall} and \texttt{exists} constraints are not restricted to the scope of relation definitions.
Since both are normal constraints, they can also be used in normal constraint generation rules.

\begin{figure*}
\begin{lstlisting}
RelationPattern.RelationPattern = <
	(<RelationDefVariant>, <RelationDefVariant>) := <Constraint>.>
VarIds = {VarId ","}*
\end{lstlisting}
\caption{The syntax proposal for NaBL2 relation definitions.}
\label{fig:nabl-syntax-proposal-relation}
\end{figure*}

\begin{figure*}
\begin{lstlisting}
Constraint.Exists  = <
  exists <VarId>@<Occurrence> in <Names> =\> <Constraint>>
Constraint.ForAll  = <
  forall <VarId>@<Occurrence> in <Names> =\> <Constraint>>
\end{lstlisting}
\caption{New constraint syntax constructs for NaBL2.}
\label{fig:nabl-syntax-proposal-constraint}
\end{figure*}

\begin{figure*}
\begin{lstlisting}
relations
  sub: Type * Type {
    (RECORD(one), RECORD(other)) :=
      forall d1@Field{x} in D(one)/Field => (
        d1 : ty1,
        exists d2@Field{x} in D(other)/Field => (
          d2 : ty2,
          ty1 <? ty2
        )
      ).
    , (List(one), List(other)) := one <? two.
  }
}
\end{lstlisting}
\caption{Example usage of the new syntax constructs for structural types.}
\label{fig:nabl-syntax-proposal-usage}
\end{figure*}

\subsection{Union and Intersection types}
We will now demonstrate that with two additional, overloaded, versions of \texttt{forall} and \texttt{exists},
NaBL2 will be powerful enough to implement union and Intersection types as described in Section \ref{sec:structural-typing-typescript}.
Figure \ref{fig:union-impl} shows how this could be implemented.
Notice that we use \texttt{forall} and \texttt{exists} in a manner similar to object types, 
but rather than operating on scopes, we now operate on lists of types.

\begin{figure*}
\begin{lstlisting}
relations
  sub: Type * Type {
    (UNION(List(types1)), UNION(List(types2))) :=
      forall t2 in types2 => (
        exists t1 in types1 => (ty1 <? ty2)
      ),
    (UNION(List(types)), ty) :=
      exists ty1 in types => (ty1 <? ty)
  }
}
\end{lstlisting}
\caption{Implementation of union types by using a custom subtype relation and overloaded \texttt{forall} and \texttt{exists} constraints.}
\label{fig:union-impl}
\end{figure*}

Intersection types are implemented with the same approach, but with the \texttt{forall} and \texttt{exists} constructs swapped to achieve behavior dual to unions.

\begin{figure*}
\begin{lstlisting}
relations
  sub: Type * Type {
    (INTERSECT(List(types1)), INTERSECT(List(types2))) :=
      forall t1 in types1 => (
        exists t2 in types2 => (ty1 <? ty2)
      ),
    (INTERSECT(List(types)), ty) :=
      forall ty1 in types => (ty1 <? ty)
  }
}
\end{lstlisting}
\caption{Intersection types, using the same constructs as union types.}
\label{fig:intersect-impl}
\end{figure*}
\section{Conclusion}

We have seen how structural types in TypeScript work as opposed to more common nominal type systems.

The definition of TypeScript and ECMAScript 2015 syntax with SDF3 is relatively straightforward
and even offers some improvements compared to the official ECMAScript 2015 specification.

Our attempt to model structural types with NaBL2 was met with more challenges.
Our efforts resulted in the insight that one pass of constraint generation and constraint solving
is not powerful enought to support structural and recursive types.

We have proposed the addition of custom relation definitions to allow language designers to define their own
subtype relations.
With this extension and two new constraints, \texttt{forall} and \texttt{exists}, language designers will 
be able to model more complex and diverse type systems.
As an example we have demonstrated that the proposed constructs can be used with Scope Graphs to model 
TypeScripts structural types, including unions and intersections.
\FloatBarrier

%% Bibliography
\bibliography{bibliography}

\end{document}
