\section{Introduction}

TypeScript\footnote{https://github.com/Microsoft/TypeScript/blob/master/doc/spec.md} is a statically typed programming language that is a superset JavaScript (ECMAScript 2015\footnote{http://www.ecma-international.org/ecma-262/6.0/}).
The aim of TypeScript is to provide large JavaScript projects with a way to integrate a static 
type system into their code in a non-intrusive way, 
while keeping the dynamic semantics equivalent to JavaScipt.
To achieve this, TypeScript compiles to human-readable JavaScript with as little transformations as possible.
Any value that has type annotations is checked by the type checker.
In addition, some type inference is done to provide static checks even for existing JavaScript without annotations.
Only when no type information can be inferred will the compiler fall back to default JavaScipt semantics.
The system of structural types enables TypeScript to be non-intrusive when used with existing JavaScript projects.
It allows programmers to gain some of benefits of static analysis while still 
being able to use existing JavaScript APIs, libraries and coding conventions.

In this report we explore how structural type systems could be implemented in terms of Scope 
Graphs with NaBL2\citep{Antwerpen:2016:CLS:2847538.2847543}.
We find that while NaBL2 already works very well for nominally typed languages with classes and inheritance,
structural types are not yet fully supported. 

Our main contribution is the proposal of a minimal set of extensions to NaBL2 that 
give language designers the possibility to define their own subtype relations. 
We show that the fundamental model of Scope Graphs is viable even for structural types, 
and that making constraint generation and resolution interleaved
allows language designers to model more complex and diverse type systems with NaBL2.

An introduction to structural type systems is given in Section \ref{sec:structural-typing}.
Then, the most significant features of the TypeScript language are discussed in Section \ref{sec:structural-typing-typescript}.
After the introduction of some of TypeScripts features, we move on to discuss, in Section \ref{sec:syntax}, how SDF3 could be used to 
easilly mirror the ECMAScript 2015 specification with the additional syntactical constructs of TypeScript.
Finally, Section \ref{sec:type-checking} discusses how structural typing can be supported in NaBL2 
as well as possible implementations of some TypeScript features in NaBL2.
