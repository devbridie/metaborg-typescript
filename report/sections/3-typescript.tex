\section{Structural Typing in Typescript}
\label{sec:structural-typing-typescript}
TypeScript was built with structural typing as it's backbone, 
nearly any feature in the language can be described in terms of structual types.
The reason for this being that the goal of the language is to provide a type system
on top of the dynamically typed JavaScript language while not introducing any 
abstractions that have a conceptually high distance to the semantics of JavaScript.
The result of this effort means that compiled TypeScript code should be suitable 
for human consumption. Structural typing provides an excellent bridge from the 
dynamically typed JavaScript world to the statically typed world of TypeScript.
In this section we will introduce the most important features of TypeScript and
how they relate to structural types.
\\
\\
Structural types manifest themselves in TypeScript as \textit{Record Types}.
In the examples above, we have already encountered some record types, a set of name-type pairs.
Built directly on top of record types are some of the most important aspects of the language:
\begin{itemize}
\item classes
\item interfaces
\item union types
\item intersection types
\end{itemize}

Other notable features include: first class and higher order functions, parametric and ad-hoc polymorphism 
(generics) and several types for special values like \texttt{null} and \texttt{undefined}. It is 
worth noting that the subtype relation within TypeScript does not form a proper latice due to 
the \texttt{any} type. Namely, a value of type \texttt{any} is a subtype of every other type, 
while every type is simultaniously a subtype of any.
\\
\\
- describe classes / interfaces in terms of sugar for records
- describe union / intersection in terms of operations on records
