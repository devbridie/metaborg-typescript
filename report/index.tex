%% For double-blind review submission
\documentclass[sigplan,10pt]{acmart}\settopmatter{printfolios=true}
%% For single-blind review submission
%\documentclass[sigplan,10pt,review]{acmart}\settopmatter{printfolios=true}
%% For final camera-ready submission
%\documentclass[sigplan,10pt]{acmart}\settopmatter{}

%% Note: Authors migrating a paper from traditional SIGPLAN
%% proceedings format to PACMPL format should change 'sigplan' to
%% 'acmsmall'.

\usepackage[T1]{fontenc}
\usepackage{lmodern}
%% Some recommended packages.
\usepackage{booktabs}   %% For formal tables:
                        %% http://ctan.org/pkg/booktabs
\usepackage{subcaption} %% For complex figures with subfigures/subcaptions
                        %% http://ctan.org/pkg/subcaption
\usepackage{listings}
\usepackage{float}
\usepackage{dblfloatfix}
\lstset{
frame = single,
basicstyle=\small
}

%% Copyright information
%% Supplied to authors (based on authors' rights management selection;
%% see authors.acm.org) by publisher for camera-ready submission
\setcopyright{none}             %% For review submission
%\setcopyright{acmcopyright}
%\setcopyright{acmlicensed}
%\setcopyright{rightsretained}
%\copyrightyear{2017}           %% If different from \acmYear


%% Bibliography style
\bibliographystyle{ACM-Reference-Format}
%% Citation style
%% Note: author/year citations are required for papers published as an
%% issue of PACMPL.
%\citestyle{acmauthoryear}  %% For author/year citations
%\citestyle{acmnumeric}     %% For numeric citations
%\setcitestyle{nosort}      %% With 'acmnumeric', to disable automatic
                            %% sorting of references within a single citation;
                            %% e.g., \cite{Smith99,Carpenter05,Baker12}
                            %% rendered as [14,5,2] rather than [2,5,14].
%\setcitesyle{nocompress}   %% With 'acmnumeric', to disable automatic
                            %% compression of sequential references within a
                            %% single citation;
                            %% e.g., \cite{Baker12,Baker14,Baker16}
                            %% rendered as [2,3,4] rather than [2-4].



\begin{document}

%% Title information
\title[Metaborg Typescript]{Implementing Typescript in Metaborg Spoofax}         %% [Short Title] is optional;
                                        %% when present, will be used in
                                        %% header instead of Full Title.
%% Author information
%% Contents and number of authors suppressed with 'anonymous'.
%% Each author should be introduced by \author, followed by
%% \authornote (optional), \orcid (optional), \affiliation, and
%% \email.
%% An author may have multiple affiliations and/or emails; repeat the
%% appropriate command.
%% Many elements are not rendered, but should be provided for metadata
%% extraction tools.

%% Author with single affiliation.
\author{Tim van der Lippe}

%% Author with two affiliations and emails.
\author{Thomas Smith}

%% Abstract
%% Note: \begin{abstract}...\end{abstract} environment must come
%% before \maketitle command
\begin{abstract}
TypeScript is a structurally typed superset of JavaScript that aims to ease the introduction of static types
to large JavaScript code bases. In this paper, we investigate how the static semantics of TypeScript can be
modeled using NaBL2. We find that while NaBL2 misses some features needed to support structural types, 
these features can be implemented in terms of the existing abstractions that NaBL2 is built on.
We show that Scope Graphs can be used to model structural types and propose a minimal set of extensions
to NaBL2 that enable language designers to define custom subtype relations. We also show that these extensions
are enough to implement most major features of the TypeScript language.
\end{abstract}

%% Keywords
%% comma separated list
\keywords{Metaborg, Spoofax, Typescript, NaBL2}  %% \keywords is optional


%% \maketitle
%% Note: \maketitle command must come after title commands, author
%% commands, abstract environment, Computing Classification System
%% environment and commands, and keywords command.
\maketitle


\section{Introduction}

TypeScript\footnote{https://github.com/Microsoft/TypeScript/blob/master/doc/spec.md} is a statically typed programming language that is a superset JavaScript (ECMAScript 2015\footnote{http://www.ecma-international.org/ecma-262/6.0/}).
The aim of TypeScript is to provide large JavaScript projects with a way to integrate a static 
type system into their code in a non-intrusive way, 
while keeping the dynamic semantics equivalent to JavaScipt.
To achieve this, TypeScript compiles to human-readable JavaScript with as little transformations as possible.
Any value that has type annotations is checked by the type checker.
In addition, some type inference is done to provide static checks even for existing JavaScript without annotations.
Only when no type information can be inferred will the compiler fall back to default JavaScipt semantics.
The system of structural types enables TypeScript to be non-intrusive when used with existing JavaScript projects.
It allows programmers to gain some of benefits of static analysis while still 
being able to use existing JavaScript APIs, libraries and coding conventions.

In this report we explore how structural type systems could be implemented in terms of Scope 
Graphs with NaBL2\citep{Antwerpen:2016:CLS:2847538.2847543}.
We find that while NaBL2 already works very well for nominally typed languages with classes and inheritance,
structural types are not yet fully supported. 

Our main contribution is the proposal of a minimal set of extensions to NaBL2 that 
give language designers the possibility to define their own subtype relations. 
We show that the fundamental model of Scope Graphs is viable even for structural types, 
and that making constraint generation and resolution interleaved
allows language designers to model more complex and diverse type systems with NaBL2.

An introduction to structural type systems is given in Section \ref{sec:structural-typing}.
Then, the most significant features of the TypeScript language are discussed in Section \ref{sec:structural-typing-typescript}.
After the introduction of some of TypeScripts features, we move on to discuss, in Section \ref{sec:syntax}, how SDF3 could be used to 
easilly mirror the ECMAScript 2015 specification with the additional syntactical constructs of TypeScript.
Finally, Section \ref{sec:type-checking} discusses how structural typing can be supported in NaBL2 
as well as possible implementations of some TypeScript features in NaBL2.

\section{Structural Typing}
\label{sec:structural-typing}

This section gives a high level overview of how structural type systems work.
Structural subtyping stands opposed to nominal subtyping. 
A majority of the most used programming languages employ a nominal type system.
In such languages, subtype relationships are explicitly declared.
The programmer has to relate two types to each other by their name,
after which the compiler will check whether the structures refered to by these names are compatible.
This process has to be done only once for each declared subtype relation.
As soon as the compiler has established that a type A is a subtype of B,
the next time that a check for this relationship has to be performed, 
a simple table lookup for the two names is sufficient.
\bigskip
With structural subtyping on the other hand, the relationships between types are implicit.
The programmer does not have to specify any relationship by hand,
every time the compiler has to check compatibility between two types, a structural comparison of the types is done.
One major consequence of this approach is that names for types become purely cosmetic.
The programmer can still assign names to types, and refer to those types by their names,
but checking whether one type is a subtype of another is completely dependent on their structure.
Generally this means that a type A is a subtype of B when the structure of A has at least as much information as the strucrure of B.
This means that a value of type A can be assigned to B by simply removing all information present in A that is not required by B.
For most languages that implement structural subtyping, the information that types carry consists of names paired with another type.
For example, in TypeScript we may write:


\begin{lstlisting}
type A = { first: string, second: string }
\end{lstlisting}

This defines the type A to be an object that contains two named fields of type 'string'.
Values of type A can be passed to anywhere a subset of \texttt{\{ first: string, second: string \}} is required.
So all the following statements are valid because the types of $x$, $y$ and $z$ are structurally subsets of $A$:

\begin{lstlisting}
const a: A =  { first: 'x', second: 'y' }

const x: { first: string } = a
const y: { second: string } = a
const z: {} = a
\end{lstlisting}

The nature of such type systems makes implementing a type checker slightly more difficult than with a nominal type system.
To demonstrate this, we consider recursive types:
\begin{lstlisting}
type Nil = {}
type Cons = { head: number, tail: NumList }
type NumList = Cons | Nil
\end{lstlisting}

The \texttt{|} operator describes that $NumList$ can either be $Cons$ nil $Nil$.
Note that $Cons$ and $NumList$ are mutually recursive.
To check whether we can assign a $NumList$ value to another value, 
we need to check if their unfoldings are equal.
Since unfolding a recursive type yields an infinite tree, 
the type checker has to make sure that this process terminates regardless of the possibly infinite type.
This approach to type checking is called \textit{equi-recursive} because a recursive type should
be a subtype of its one step unfolding.\citep{pierce2002types}

\section{Structural Typing in Typescript}
\label{sec:structural-typing-typescript}
TypeScript was built with structural typing as it's backbone, 
nearly any feature in the language can be described in terms of structual types.
The reason for this being that the goal of the language is to provide a type system
on top of the dynamically typed JavaScript language while not introducing any 
abstractions that have a conceptually high distance to the semantics of JavaScript.
The result of this effort means that compiled TypeScript code should be suitable 
for human consumption. Structural typing provides an excellent bridge from the 
dynamically typed JavaScript world to the statically typed world of TypeScript.
In this section we will introduce the most important features of TypeScript and
how they relate to structural types.
\bigskip
Structural types manifest themselves in TypeScript as \textit{Record Types}.
In the examples above, we have already encountered some record types, a set of name-type pairs.
Built directly on top of record types are some of the most important aspects of the language:
\begin{itemize}
\item classes
\item interfaces
\item union types
\item intersection types
\end{itemize}

Other notable features include: first class and higher order functions, parametric and ad-hoc polymorphism 
(generics) and several types for special values like \texttt{null} and \texttt{undefined}. It is 
worth noting that the subtype relation within TypeScript does not form a proper latice due to 
the \texttt{any} type. Namely, a value of type \texttt{any} is a subtype of every other type, 
while every type is simultaniously a subtype of any.
\bigskip
\subsection{Classes and Interfaces}
While syntax for defining classes in TypeScript is similar to Java, the structural typing makes classes semantically different form nominally typed languages.
The following snippet defines a simple class in TypeScript:

\begin{lstlisting}
class NonEmptyList<A> {
  head: A;
  tail: List<A>;
  constructor(head: A) {
    this.head = head;
    this.tail = new List();
  }
}
\end{lstlisting}

In any nominally typed language, it would not be possible to assign a $NonEmptyList$ to any variable that is not also a $NonEmptyList$.
In TypeScript this restriction does not apply. 
Any instance of a class can be assigned a variable with any type as long as the two are structurally in a subtype relationship.
This means that the following snippet is perfectly legal:

\begin{lstlisting}
class SignletonList<A> {
  head: A;
}

const s: SingletonList<number> = 
  new NonEmptyList(42);
\end{lstlisting}

Since names are irrelevant to the type checker, the following is also valid because the type of 
$s$ is structurally equivalent to $NonEmptyList<number>$.
\bigskip
\begin{lstlisting}
const s: { head: number, tail: List<number> } =
  new NonEmptyList(42);
\end{lstlisting}

This situation remains unchanged when we add 'methods' to the $NonEmptyList$ class:

\begin{lstlisting}
class NonEmptyList<A> {
  head: A;
  tail: List<A>;
  ...
  size(): number {
    return 1 + tail.size();
  }
}

const s: { 
  head: number, tail: List<number>,
  size: () => number  
} = new NonEmptyList(42);
\end{lstlisting}

From the above snippet we can see that classes are simply sugar for named object types.
The fields are directly translated to object members, and class methods are translated
to object members that have a function type corresponding to the method signature.
A consequence of this is that class methods are first class values.
Class constructors are translated to a special type of function outside the class scope that has the exact name of the class:

\begin{lstlisting}
class NonEmptyList<A> {
  ...
  constructor(head: A) {
    ...
  }
}
const a: typeof NonEmptyList = NonEmptyList;
const xs: NonEmptyList<number> = new a(42);
\end{lstlisting}

From this it appears that classes live in both the value namespace as well as the type namespace,
because an instance of $NonEmptyList$ ($xs$) can be created from a runtime value ($a$) on the last line.
In reality, $a$ represents the constructor for $NonEmptyList$ rather than the class itself.
We can therefore conclude that class constructors are first class values just like class methods.
The type of a constructor must different from the type of a function because the former can only be called with the $new$ keyword.
The $typeof$ keyword in the definition of $a$ is used in this case to signal that $a$ is a constructor.
Since constructors do not live inside the class scope, they have no impact type checking when handling instances of a class.
\bigskip
Inheritance is also handled in a straightforward manner.
When checking whether an instance of a class can be assigned to an object type, the compiler
simply takes all class members and all members of any parent class into account.
The rules that apply to classes can similarly be applied to interfaces, with the exception that interfaces do not have a constructor.

\subsection{Union and Intersection types}
Union and intersection types provide for a way to compose types, similar to sum and product types in many functional languages.
\begin{lstlisting}
interface X = { x: number }
interface Y = { y: string }
type U = X | Y
\end{lstlisting}

The last line in the code above defines $U$ to be the union of $X$ and $Y$. 
This means that $U$ can be either $X$ or $Y$ but not both, similar to discriminated unions in other languages.
Union types have the following subtype relationships:
\begin{itemize}
\item A union type $U$ is a subtype of a type $T$ if each type in $U$ is a subtype of $T$.
\item A type $T$ is a subtype of a union type $U$ if $T$ is a subtype of any type in $U$.
\end{itemize}
Similarly, union types have the following assignability relationships:
\begin{itemize}
\item A union type $U$ is assignable to a type $T$ if each type in $U$ is assignable to $T$.
\item A type $T$ is assignable to a union type $U$ if $T$ is assignable to any type in $U$.
\end{itemize}

When the types that make up the union have overlapping members, those members themselves become unions:
\begin{lstlisting}
interface X = { q: number }
interface Y = { q: string }
type U = X | Y
// U is equivalent to { q: number | string }
\end{lstlisting}
\bigskip
Intersection types describe values with multiple simultanious types. 
\begin{lstlisting}
interface X = { x: number }
interface Y = { y: string }
type I = X & Y
// I is equivalent to { x: number, y: string }
\end{lstlisting}

Just as with unions, when the types that make up an intersection have overlapping members, those members themselves become intersections:
\begin{lstlisting}
interface X = { q: number }
interface Y = { q: string }
type I = X & Y
// I is equivalent to { q: number & string }
\end{lstlisting}

Intersection types have the following subtype relationships:
\begin{itemize}
\item An intersection type $I$ is a subtype of a type $T$ if any type in $I$ is a subtype of $T$.
\item A type $T$ is a subtype of an intersection type $I$ if $T$ is a subtype of each type in $I$.
\end{itemize}
Similarly, intersection types have the following assignability relationships:
\begin{itemize}
\item An intersection type $I$ is assignable to a type $T$ if any type in $I$ is assignable to $T$.
\item A type $T$ is assignable to an intersection type $I$ if $T$ is assignable to each type in $I$.
\end{itemize}
\bigskip
Since functions are values in TypeScript, we can even define unions and intersections over functions.
The following example demonstrates the intersting case of an intersection over functions:
\begin{lstlisting}
type F1 = (a: string, b: string) => void;  
type F2 = (a: number, b: number) => void;

var f: F1 & F2 = (a: string | number, b: string | number) => { };  
f("hello", "world");
f(1, 2);
f(1, "test");
\end{lstlisting}

The last call to $f$ will be rejected by the compiler.
The type signature of $f$ makes it possible for the type checker to infer
that this function should only be called with two strings or two numbers.
If the type signature were removed, this program would be well-typed.
\bigskip
In the next sections we will discuss how these features would ideally be implemented in Spoofax.


\section{Syntax definition}
The syntax definition of Typescript is defined in the TypeScript Language Specification\footnote{https://github.com/Microsoft/TypeScript/blob/master/doc/spec.md}.
Eventhough this specification is not up-to-date for TypeScript 2, the syntax has not significantly changed between TypeScript 1 and 2.
The specification builds upon the ECMAScript® 2015 Language Specification\footnote{http://www.ecma-international.org/ecma-262/6.0/}, as TypeScript is primarily a superset of ECMAScript.
While it is defined as a superset with additions to the syntax definitions, several productions in the TypeScript grammar modify/replace the productions with the same name in the ECMAScript definition.

% The translation of the syntax definition into the SDF3\citep{Vollebregt:2012:DST:2427048.2427056} productions started

\section{Type Checking}
\label{sec:type-checking}

\input{sections/6-recommendations}
\section{Conclusion}

We have seen how structural types in TypeScript work as opposed to more common nominal type systems.

The definition of TypeScript and ECMAScript 2015 syntax with SDF3 is relatively straightforward
and even offers some improvements compared to the official ECMAScript 2015 specification.

Our attempt to model structural types with NaBL2 was met with more challenges.
Our efforts resulted in the insight that one pass of constraint generation and constraint solving
is not powerful enought to support structural and recursive types.

We have proposed the addition of custom relation definitions to allow language designers to define their own
subtype relations.
With this extension and two new constraints, \texttt{forall} and \texttt{exists}, language designers will 
be able to model more complex and diverse type systems.
As an example we have demonstrated that the proposed constructs can be used with Scope Graphs to model 
TypeScripts structural types, including unions and intersections.
\FloatBarrier

%% Bibliography
\bibliography{bibliography}

\end{document}
